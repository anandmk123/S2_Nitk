\documentclass{beamer}
\usepackage{adjustbox}
\usepackage{algorithm}
\usepackage{algpseudocode}
\usepackage{booktabs}

% Use the Madrid theme for a clean and professional look
\usetheme{Madrid}

% Title and Author information
\title{Career Recommendation System for 10th Grade Students}
\subtitle{}
\author{
	\textbf{ANAND M K} \\ % Author
	Roll Number: 242CS008 \\ % Roll number
}
\institute{
	Department of Computer Science and Engineering \\ 
	National Institute of Technology Karnataka (NITK) \\ 
	Surathkal, India
}

\date{} % Remove the date or keep it empty

% Remove the footer that displays names
\setbeamertemplate{footline}{}

% Change the color of the title text to white
\setbeamercolor{title}{fg=white}


\begin{document}

\begin{frame}
    \titlepage
\end{frame}


\begin{frame}{Objective}
    \begin{itemize}
        \item To develop a personalized career recommendation system for 10th-grade students.
        \item The system considers:
        \begin{itemize}
            \item Academic performance in subjects like:
            Physics, Chemistry, Mathematics, Biology, English, Social Science.
            \item Personality traits and personal interests.
        \end{itemize}
        \item Recommendations include:
        \begin{itemize}
            \item Career options.
            \item +2 stream.
            \item Degree programs.
        \end{itemize}
    \end{itemize}
\end{frame}

\begin{frame}{Data Collection}
    \begin{itemize}
        \item Gather data on:
        \begin{itemize}
            \item \textbf{Academic performance}: Collect subject-wise scores in:
            \begin{itemize}
                \item Physics, Chemistry, Mathematics, Biology, English, and Social Science.
                \item Focus on identifying strengths in specific subjects.
            \end{itemize}
            \item \textbf{Personality traits}: Use psychometric tests like:
            \begin{itemize}
                \item The Big Five personality test (measures openness, conscientiousness, etc.).
                \item Holland Codes (RIASEC model for career-oriented traits).
            \end{itemize}
            \item \textbf{Personal interests}: Design surveys or questionnaires to identify:
            \begin{itemize}
                \item Hobbies and extracurricular activities (e.g., coding, writing, sports).
            \end{itemize}
        \end{itemize}
    \end{itemize}
\end{frame}

\begin{frame}{Content-Based Filtering for Career Recommendation}
    \begin{itemize}
        \item Content-based filtering recommends careers to students based on the features of both students and careers.
        \item \textbf{How It Works:}
        \begin{itemize}
            \item Career profiles are created by defining the features such as:
            \begin{itemize}
                \item Academic requirements (subject marks).
                \item Personality traits (e.g., openness, conscientiousness).
                \item Personal interests (e.g., hobbies, career aspirations).
            \end{itemize}
            \item Student profiles are collected based on:
            \begin{itemize}
                \item Marks in various subjects.
                \item Personality traits (from psychometric tests).
                \item Interests (from surveys or questionnaires).
            \end{itemize}
            \item The system calculates the similarity between student profiles and career profiles using similarity metrics.
        \end{itemize}
    \end{itemize}
\end{frame}

\begin{frame}{Steps in Content-Based Filtering}
    \begin{itemize}
        \item \textbf{Step 1: Create Career Profiles:}
        \begin{itemize}
            \item Define career features based on subject requirements, personality traits, and interests.
            \item Example: Software Engineer career profile.
        \end{itemize}
        \item \textbf{Step 2: Collect Student Data:}
        \begin{itemize}
            \item Collect marks in subjects like Physics, Maths, Chemistry, Biology.
            \item Collect personality traits and interests.
        \end{itemize}
        \item \textbf{Step 3: Similarity Calculation:}
        \begin{itemize}
            \item Calculate the similarity between the student's profile and each career profile.
            \item Use similarity metrics such as cosine similarity or Euclidean distance.
        \end{itemize}
        \item \textbf{Step 4: Recommend Top Careers:}
        \begin{itemize}
            \item Rank careers based on similarity scores.
            \item Recommend the top careers to the student.
        \end{itemize}
    \end{itemize}
\end{frame}

\begin{frame}{Career Profile Example}
    \begin{itemize}
        \item \textbf{Career Profile: Software Engineer}
        \begin{itemize}
            \item \textbf{Subject Requirements:} High scores in Maths.
            \item \textbf{Personality Traits:} High analytical thinking, problem-solving skills, and conscientiousness.
            \item \textbf{Interests:} Coding, software development, AI, and technology.
        \end{itemize}
        \item \textbf{Student Profile Example:}
        \begin{itemize}
            \item \textbf{Marks:} Maths = 85, Physics = 80, Chemistry = 70.
            \item \textbf{Personality Traits:} Analytical, problem-solving, detail-oriented.
            \item \textbf{Interests:} Interest in programming and technology.
        \end{itemize}
    \end{itemize}
\end{frame}

\begin{frame}{Similarity Calculation}
    \begin{itemize}
        \item The system computes the similarity between the student's profile and the career profile using cosine similarity.
        \item \textbf{Cosine Similarity Formula:}
        \[
        \text{Similarity} = \frac{\vec{A} \cdot \vec{B}}{|\vec{A}| \cdot |\vec{B}|}
        \]
        Where \( \vec{A} \) and \( \vec{B} \) are the student and career feature vectors, respectively.
        \item \textbf{Example:}
        \begin{itemize}
            \item Student Profile: [85, 80, 70, 0.8, 0.7, 0.6] (marks + personality traits).
            \item Career Profile: [80, 75, 60, 0.9, 0.8, 0.7] (subject requirements + personality traits).
        \end{itemize}
    \end{itemize}
\end{frame}

\begin{frame}{Recommendation Output}
    \begin{itemize}
        \item The system computes similarity scores for all career profiles and ranks them.
        \item The top careers are recommended to the student.
        \item \textbf{Example Output:}
        \begin{enumerate}
            \item \textbf{Software Engineer:} Similarity score = 0.85.
            \item \textbf{Data Scientist:} Similarity score = 0.80.
            \item \textbf{Mechanical Engineer:} Similarity score = 0.75.
        \end{enumerate}
        \item The career with the highest similarity is recommended as the top career.
    \end{itemize}
\end{frame}

\begin{frame}{Career Path Recommendation (Part 1)}
    \begin{itemize}
        \item
        Guide students on the educational path required to achieve their chosen career.

        \item \textbf{Components of the Pathway:}
        \begin{itemize}
            \item \textbf{+2 Stream:}
            \item Suggest the ideal stream for Grade 12 based on the chosen career.
            \item Examples:
            \begin{itemize}
                \item \textbf{Software Engineer:} Science Stream with Maths and Computer Science.
                \item \textbf{Doctor:} Science Stream with Biology, Chemistry, and Physics.
            \end{itemize}
        \end{itemize}
    \end{itemize}
\end{frame}

\begin{frame}{Career Path Recommendation (Part 2)}
    \begin{itemize}
        \item \textbf{Undergraduate Degree:}
        Recommend degree programs based on the chosen career. Examples:
        \begin{itemize}
            \item \textbf{Software Engineer:} B.Tech in Computer Science or IT.
            \item \textbf{Doctor:} MBBS, followed by specialization.
        \end{itemize}

        \item \textbf{Additional Requirements:}
        Highlight any entrance exams or certifications. Examples:
        \begin{itemize}
            \item \textbf{Software Engineer:} Coding bootcamps, internships at tech companies.
            \item \textbf{Doctor:} NEET exam for medical college admission.
        \end{itemize}

        \item \textbf{Output: Career Path:}
        Example for Software Engineer:
        \begin{enumerate}
            \item +2 Stream: Science with Maths and Computer Science.
            \item Degree: B.Tech in Computer Science or IT.
            \item Additional: Competitive programming, internships, certifications.
        \end{enumerate}
    \end{itemize}
\end{frame}

\begin{frame}
\centering
\Huge
\textbf{Thank You!}

\vspace{2cm}
\end{frame}


\end{document}

