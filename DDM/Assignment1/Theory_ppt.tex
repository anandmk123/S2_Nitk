\documentclass{beamer}

\usepackage[utf8]{inputenc}
\usepackage{graphicx}
\usepackage{booktabs}
\usepackage{amssymb}
\usepackage{setspace}
\usepackage{titlesec}

% Title and section formatting
\titleformat{\section}{\large\bfseries}{\thesection.}{1em}{}
\titleformat{\subsection}{\normalsize\bfseries}{\thesubsection.}{1em}{}

% Beamer theme
\usetheme{Madrid}
\usecolortheme{default}
\setbeamertemplate{footline}{}
% Title slide
\title{A Systematic Literature Review on Datasets for Deepfake Images in Smart Cities}
\author{ANAND M K \\ National Institute of Technology Karnataka, Surathkal \\ \texttt{242CS008}}
\date{}

\begin{document}

% Title slide
\begin{frame}
    \titlepage
\end{frame}

% Abstract slide
\begin{frame}
    \frametitle{Abstract}
    \begin{itemize}
        \item Deepfake image datasets are critical for developing robust detection systems in smart cities.
        \item This review systematically examines:
            \begin{itemize}
                \item Key datasets: DFDC, DeeperForensics, FaceForensics++, DeepfakeTIMIT, UADFV.
                \item Deepfake generation techniques: GANs, Autoencoders, and hybrid models.
                \item Preprocessing methods: Data augmentation, feature extraction, artifact analysis.
                \item Detection methods: CNNs, RNNs, and hybrid models.
            \end{itemize}
        \item Challenges: Dataset biases, generalization issues, real-time processing requirements.
    \end{itemize}
\end{frame}

% Introduction slide
\begin{frame}
    \frametitle{Introduction}
    \begin{itemize}
        \item \textbf{Deepfake Technology:}
            \begin{itemize}
                \item Rapid advancements in AI and deep learning enable the creation of highly realistic synthetic images and videos.
                \item Applications: Entertainment, education, accessibility.
                \item Risks: Identity theft, misinformation, unauthorized access.
            \end{itemize}
        \item \textbf{Smart Cities:}
            \begin{itemize}
                \item Rely on IoT, AI, and big data analytics for urban infrastructure and services.
                \item Vulnerabilities: Surveillance systems, authentication protocols, public information dissemination.
            \end{itemize}
        \item \textbf{Need for Detection:}
            \begin{itemize}
                \item High-quality datasets are essential for training and evaluating detection models.
                \item Current datasets: DFDC, FaceForensics++, DeeperForensics-1.0, DF-TIMIT, UADFV.
            \end{itemize}
    \end{itemize}
\end{frame}

% Method slide
\begin{frame}
    \frametitle{Method}
    \begin{itemize}
        \item \textbf{Literature Search Procedure:}
            \begin{itemize}
                \item Databases: ScienceDirect, IEEE Xplore, ACM Digital Library.
                \item Keywords: "deepfake detection," "image dataset," "smart city surveillance."
            \end{itemize}
        \item \textbf{Research Problems:}
            \begin{itemize}
                \item Focus on smart city security, deepfake detection, and dataset creation methodologies.
                \item Inclusion criteria: Experimental studies, surveys, observational studies.
            \end{itemize}
        \item \textbf{Search Strategy:}
            \begin{itemize}
                \item Boolean operators: AND, OR.
                \item Search string: "deepfake detection" AND "image dataset" AND "smart city surveillance."
            \end{itemize}
        \item \textbf{Selection of Studies:}
            \begin{itemize}
                \item Data extraction: Authors, publication year, dataset details, detection methods.
                \item Analysis: Comparative evaluation of datasets and detection techniques.
            \end{itemize}
    \end{itemize}
\end{frame}

% DFDC Dataset slide
\begin{frame}
    \frametitle{DeepFake Detection Challenge (DFDC) Dataset}
    \begin{itemize}
        \item \textbf{Overview:}
            \begin{itemize}
                \item Largest and most diverse deepfake video dataset.
                \item Contains over 100,000 video clips from 3,426 actors.
            \end{itemize}
        \item \textbf{Generation Methods:}
            \begin{itemize}
                \item Deepfake Autoencoder (DFAE): Shared encoder with two decoders.
                \item MM/NN Face Swap: Custom morphable-mask model.
                \item Neural Talking Heads (NTH): Meta-learning approach.
                \item FSGAN: GAN-based face swapping and reenactment.
            \end{itemize}
        \item \textbf{Augmentations:}
            \begin{itemize}
                \item Distractors: Overlay objects like images, shapes, and text.
                \item Augmenters: Geometric, color, and framerate transformations.
            \end{itemize}
        \item \textbf{Detection Methods:}
            \begin{itemize}
                \item Efficient ViT, Convolutional Cross ViT.
                \item Pre-extraction of faces using MTCNN.
            \end{itemize}
    \end{itemize}
\end{frame}

% FaceForensics++ Dataset slide
\begin{frame}
    \frametitle{FaceForensics++ Dataset}
    \begin{itemize}
        \item \textbf{Overview:}
            \begin{itemize}
                \item Standardized benchmark for facial manipulation detection.
                \item Contains 5,000 videos with real and manipulated content.
            \end{itemize}
        \item \textbf{Manipulation Methods:}
            \begin{itemize}
                \item FaceSwap: Graphics-based face region transfer.
                \item DeepFakes: Neural network-based face replacement.
                \item Face2Face: Facial reenactment.
                \item NeuralTextures: Neural texture-based rendering.
            \end{itemize}
        \item \textbf{Postprocessing:}
            \begin{itemize}
                \item Simulates video quality degradation using H.264 compression.
                \item Two levels: High-quality (HQ) and low-quality (LQ).
            \end{itemize}
        \item \textbf{Detection Methods:}
            \begin{itemize}
                \item Handcrafted features: Steganalysis with SVM.
                \item Learned features: CNNs, XceptionNet, MesoInception-4.
            \end{itemize}
    \end{itemize}
\end{frame}

% DeeperForensics-1.0 Dataset slide
\begin{frame}
    \frametitle{DeeperForensics-1.0 Dataset}
    \begin{itemize}
        \item \textbf{Overview:}
            \begin{itemize}
                \item Largest face forgery detection dataset: 60,000 videos, 17.6 million frames.
                \item Focus on real-world perturbations for robustness.
            \end{itemize}
        \item \textbf{Generation Framework:}
            \begin{itemize}
                \item DeepFake Variational Auto-Encoder (DF-VAE).
                \item Disentangles structure (expression, pose) from appearance (texture, skin color).
                \item Masked Adaptive Instance Normalization (MAdaIN) for style mismatch.
            \end{itemize}
        \item \textbf{Diversity:}
            \begin{itemize}
                \item 100 actors with varied genders, ages, skin tones, and nationalities.
                \item Professional indoor setting with diverse lighting and camera perspectives.
            \end{itemize}
        \item \textbf{Detection Methods:}
            \begin{itemize}
                \item Image-level: Xception-Net.
                \item Video-level: C3D, TSN.
            \end{itemize}
    \end{itemize}
\end{frame}

% DF-TIMIT Dataset slide
\begin{frame}
    \frametitle{DF-TIMIT Dataset}
    \begin{itemize}
        \item \textbf{Overview:}
            \begin{itemize}
                \item Created using GANs on VidTIMIT database.
                \item Contains 640 videos (low and high quality).
            \end{itemize}
        \item \textbf{Generation Process:}
            \begin{itemize}
                \item Low-quality (LQ): 64x64 face regions, 200 frames at 4 fps.
                \item High-quality (HQ): 128x128 face regions, 400 frames at 8 fps.
                \item Blending techniques: CNN-based segmentation (LQ), landmark alignment (HQ).
            \end{itemize}
        \item \textbf{Detection Methods:}
            \begin{itemize}
                \item Lip-syncing detection: MFCCs, LSTM.
                \item Image-based systems: PCA+LDA, IQM+SVM.
            \end{itemize}
    \end{itemize}
\end{frame}

% UADFV Dataset slide
\begin{frame}
    \frametitle{UADFV Dataset}
    \begin{itemize}
        \item \textbf{Overview:}
            \begin{itemize}
                \item Comprises 49 real and 49 Deepfake videos.
                \item Average duration: 11.14 seconds, resolution: 294×500 pixels.
            \end{itemize}
        \item \textbf{Detection Method:}
            \begin{itemize}
                \item 3D head pose inconsistencies.
                \item Features extracted using DLib and OpenFace2.
                \item SVM classifiers for final classification.
            \end{itemize}
    \end{itemize}
\end{frame}

% Quantitative Comparison slide
\begin{frame}
    \frametitle{Quantitative Comparison of Deepfake Datasets}
    \begin{table}[ht]
        \centering
        \caption{Quantitative comparison of various Deepfake datasets}
        \begin{tabular}{lccc}
            \toprule
            \textbf{Dataset} & \textbf{Unique fake videos} & \textbf{Total videos} & \textbf{Total subjects} \\ 
            \midrule
            DF-TIMIT         & 640     & 960     & 43   \\
            UADFV            & 49      & 49      & 49   \\
            FF++ DF          & 4,000   & 5,000   & ?    \\
            DeeperForensics-1.0 & 1,000 & 60,000  & 100  \\
            DFDC             & 104,500 & 128,154 & 960  \\
            \bottomrule
        \end{tabular}
        \label{tab:deepfake_comparison}
    \end{table}
\end{frame}

% Conclusion slide
\begin{frame}
    \frametitle{Conclusion}
    \begin{itemize}
        \item \textbf{Dataset Evolution:}
            \begin{itemize}
                \item Early datasets: UADFV, DF-TIMIT (small-scale, limited diversity).
                \item Second-generation: FaceForensics++ (higher quality, ethical considerations).
                \item Third-generation: DeeperForensics-1.0, DFDC (large-scale, diverse, robust).
            \end{itemize}
        \item \textbf{Challenges:}
            \begin{itemize}
                \item Dataset biases, generalization issues, real-time processing.
            \end{itemize}
        \item \textbf{Future Directions:}
            \begin{itemize}
                \item Improve dataset diversity and ethical compliance.
                \item Develop scalable and real-time detection systems.
            \end{itemize}
    \end{itemize}
\end{frame}




\end{document}