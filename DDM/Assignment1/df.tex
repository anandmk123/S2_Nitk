\documentclass[a4paper]{article}
\usepackage[utf8]{inputenc}
\usepackage[margin=1in]{geometry}
\usepackage{graphicx}
\usepackage{setspace}
\usepackage{titlesec}

% Title and section formatting
\titleformat{\section}{\large\bfseries}{\thesection.}{1em}{}
\titleformat{\subsection}{\normalsize\bfseries}{\thesubsection.}{1em}{}

% Line spacing
\doublespacing

% Begin document
\begin{document}

% Title
\begin{center}
    \Large \textbf{A Systematic Literature Review on Datasets for Deepfake Images in Smart Cities}
\end{center}

% Authors and affiliations
\begin{center}
    \normalsize
    ANAND M K \\
    National Institute of Technology Karnataka, Surathkal \\
    anandmk.242cs008@nitk.edu.in
\end{center}

% Abstract
\begin{abstract}
Handling deepfake image datasets is increasingly relevant for the advancement of smart city environments. This systematic literature review examines state-of-the-art datasets tailored to deepfake image creation and detection, highlighting their use in urban applications such as surveillance, fraud prevention, and misinformation control. By analyzing recent studies, this review explores current research trends, key datasets, and challenges in implementing deepfake technologies in smart cities. The study provides insights into dataset quality, scalability, and applicability while identifying critical gaps for future research in this domain.
\end{abstract}

% Keywords
\noindent \textbf{Keywords:} Deepfake datasets, Smart cities, Surveillance, Misinformation, Urban technology.

\end{document}
